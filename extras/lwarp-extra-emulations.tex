\makeatletter

% lwarp emula comandos de floatrow, mas não os comandos de float
% que existem em floatrow; aqui, incluímos a emulação de float
\usepackage{float}

% Este comando de floatrow não é emulado, mas nós utilizamos,
% então emulamos aqui
\providecommand{\DeclareFloatVCode}[2]{}

% Por padrão ele faz "blockquotation", que pelo jeito não existe no HTML.
% Funciona no browser porque ele cria uma regra CSS correspondente, mas
% o pandoc e o libreoffice, ao importar, ignoram a formatação. Aqui,
% repetimos o código de lwarp para "quote".
\renewenvironment*{quotation}
  {
    \LWR@forcenewpage
    \LWR@htmlblocktag{blockquote}
  }
  {\LWR@htmlblocktag{/blockquote}}

% Modifica verse em lwarp para permitir o asterisco em \\*
\renewenvironment{verse}
  {
    \newcommand{\@LWRNELnewline}{\@ifstar{\newline}{\newline}}
    \let\\\@LWRNELnewline
    \list{}{\itemsep      \z@
            \itemindent   -1.5em%
            \listparindent\itemindent
            \rightmargin  \leftmargin
            \advance\leftmargin 1.5em}%
    \item\relax}
  {\endlist}

% Hyperlinks de/para as notas de rodapé
% O identificador das notas de rodapé nos apêndices e nos anexos é
% o mesmo (...fn.A-1, onde A é o "número" do apêndice ou anexo).
% Seria bom acrescentar \@chapapp ou algo assim para contornar isso.
\ifcsdef{chapter}
  {
    \long\def\@makefntext#1{%
        \hypertarget{fn.\thechapter-\@thefnmark}{}%
	\hyperlink{fnbk.\thechapter-\@thefnmark}{\textsuperscript{\@thefnmark}}%
        ~#1%
    }

    \def\@makefnmark{%
        \hypertarget{fnbk.\thechapter-\@thefnmark}{}%
	\hyperlink{fn.\thechapter-\@thefnmark}{\textsuperscript{\@thefnmark}}%
    }
  }
  {
    \long\def\@makefntext#1{%
        \hypertarget{fn.\thesection-\@thefnmark}{}%
	\hyperlink{fnbk.\thesection-\@thefnmark}{\textsuperscript{\@thefnmark}}%
        ~#1%
    }

    \def\@makefnmark{%
        \hypertarget{fnbk.\thesection-\@thefnmark}{}%
	\hyperlink{fn.\thesection-\@thefnmark}{\textsuperscript{\@thefnmark}}%
    }
  }

% lwarp defines \quad as html code "&#x2003;". If "FormatWP" is true,
% \hspace is defined as a sequence of \quad's that approximate the
% dimension requested. If this dimension is large, however, the text
% generated (&#x2003;&#x2003;&#x2003;...) may be too large to fit
% in a line when the "fake" pdf is generated by lwarp (it becomes
% an overfull hbox). If this happens, the html output becomes incorrect.
% We would like LaTeX to be able to break the line, but for that it
% needs some space characters. We cannot simply add spaces or line
% breaks between successive \quad's because this would modify the size
% of the \hspace. What we do, then, is add comments: <!-- -->. But
% since we also do not want to generate an enormous amount of lines
% (because this might make the text longer than the pdf page), we add
% a comment like this only every 7 \quads. Note that we might add
% such a comment in places where there are no sequence of \quad's,
% but this is simple and causes no significant problems. We might
% change this slightly so that we only do this inside \hspace.

\newcounter{quads}
\appto{\quad}{%
  \ifnumgreater{\thequads}{7}{\setcounter{quads}{0}<!-- -->}{\stepcounter{quads}}%
}

\makeatother
