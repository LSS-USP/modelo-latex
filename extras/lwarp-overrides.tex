\makeatletter

% O comando precisa existir para podermos fazer \renewcommand; como
% \poemtitle é definido por titlesec e lwarp desabilita todos os
% comandos de titlesec, precisamos fazer o comando existir.
%\providecommand{\poemtitle}{\relax}
%\renewcommand{\poemtitle}{\@ifstar{\realpoemtitle}{\realpoemtitle}}

%\newcommand{\realpoemtitle}[1]{\hspace{1.5em}\textbf{#1}\par\vspace{\parskip}}
%\renewcommand{\poemlist}[1]{\relax}

% lwarp's microtype emulation does something weird with this command;
% we cannot load lwarp-letterspace because it clashes with microtype;
% so, we just correct this here.
\DeclareDocumentCommand{\textls}{o +m}{#2}

\newlength\MKwidth%
\newlength\MKheight%
\renewcommand{\makecell}[2][]{%
\settowidth{\MKwidth}{#2}%
\parbox[#1]{\MKwidth}{#2}%
}

% listings: lwarp faz parskip=0, mas isso faz a primeira linha
% (onde vai o <pre>) ser escrita em cima da primeira linha do programa
% no pdf; com isso, o texto é extraído erroneamente. Aqui, acrescentamos
% uma linha
\xpatchcmd{\LWR@atbeginverbatim}{\LWR@orignewline}{\LWR@orignewline\LWR@orignewline}{}{}

% lwarp insere quebras de linha ao longo do texto (html) gerado para
% garantir que não haja overflow na vertical. Normalmente isso não é
% um problema, mas com verbatim (e, portanto, com listings), pode
% acontecer isto:
%
% <pre>
%     linha 1
% ----> quebra de página
%     linha 2
% ----> quebra de página
%     linha 3
% </pre>
%
% O problema é que, na segunda página, não há nada impresso além da
% linha 2; por conta disso, pdftotext não "percebe" que essa linha
% está indentada, o que faz código-fonte com listings aparecer com
% erros de indentação. O certo seria sempre "gobble" os espaços,
% mas aparentemente a reimplementação de lwarp para listings
% ignora esse parâmetro. Então, não sei como resolver isso.

% Outro problema com listings é que a numeração das linhas de
% código processa não as linhas de entrada, mas as linhas de
% saída, ou seja, o código html. Como esse código inclui alguns
% comentários (<!-- ... -->), há linhas a mais, gerando números
% errados. Para piorar, alguns desses números ficam dentro do
% comentário html, então o resultado visual é que há "pulos"
% nos números de linha. Também não sei como resolver, mas nem
% comecei a olhar as possibilidades.

% Não precisamos de ruledcaption; aqui, fazemos o comando simplesmente
% chamar o float program.
\renewenvironment{ruledcaption}[2][]{
    \begin{program}
        \ifstrempty{#1}
            {\caption{#2}}%
            {\caption[#1]{#2}}%
}
{\end{program}}

\makeatother
