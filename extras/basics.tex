%%%%%%%%%%%%%%%%%%%%%%%%%%%%%%%%%%%%%%%%%%%%%%%%%%%%%%%%%%%%%%%%%%%%%%%%%%%%%%%%
%%%%%%%%%%%%%%%%%%%%%%% CONFIGURAÇÕES E PACOTES BÁSICOS %%%%%%%%%%%%%%%%%%%%%%%%
%%%%%%%%%%%%%%%%%%%%%%%%%%%%%%%%%%%%%%%%%%%%%%%%%%%%%%%%%%%%%%%%%%%%%%%%%%%%%%%%

% Vários comandos auxiliares para o desenvolvimento de packages e classes;
% aqui, usamos em alguns comandos de formatação e condicionais.
\usepackage{etoolbox}
\usepackage{xstring}
\usepackage{expl3}
\usepackage{xparse}
\usepackage{letltxmacro}
\usepackage{regexpatch}
% Algumas packages dependem de xpatch e tentam carregá-la, causando conflitos
% com regexpatch. Como regexpatch oferece todos os recursos de xpatch (ela
% é uma versão estendida de xpatch, mas ainda considerada experimental), vamos
% fazê-las acreditar que xpatch já foi carregada.
\expandafter\xdef\csname ver@xpatch.sty\endcsname{2012/10/02}

% Arithmetic expressions in \set{length,counter} & \addto{length,counter};
% commands \widthof, \heightof, \depthof, \totalheightof, \settototalheight
\usepackage{calc}

% Sempre que possível, é melhor usar os recursos de etoolbox ao invés de
% ifthen; no entanto, várias packages dependem dela.
%\usepackage{ifthen}

% Esta não está em uso mas pode ser útil.
%\usepackage{ltxcmds}

%\usepackage{xfp} % Floating-point calculations

% Esta package permite detectar XeTeX, LuaTeX e pdfTeX, mas pode não estar
% disponível em todas as instalações de TeX.
%\usepackage{iftex}
% Por conta disso, usaremos estas (que não detectam pdfTeX):
\usepackage{ifxetex}
\usepackage{ifluatex}

\newbool{unicodeengine}
\ifboolexpr{bool{xetex} or bool{luatex}}
  {\booltrue{unicodeengine}}
  {\boolfalse{unicodeengine}}

% Detecta se estamos produzindo um arquivo PDF ou DVI (lembrando que tanto
% pdfTeX quanto LuaTeX podem gerar ambos)
\usepackage{ifpdf}

% lwarp deveria ser carregada depois de fontspec, unicode-math,
% fontaxes etc. mas não dá: as packages da ams precisam ser
% carregadas *antes* dessas e lwarp deve ser carregada antes
% das packages da ams...
% Como lwarp é um tanto guloso, isto pode ser necessário
%\usepackage{morewrites}
%\morewritessetup{allocate=10}

\usepackage[latexmk]{lwarp}
\booltrue{FormatWP}
\boolfalse{WPMarkFloats}
\boolfalse{WPMarkMinipages}
\boolfalse{WPMarkTOC}
\booltrue{WPTitleHeading}
% Adições a lwarp, carregadas imediatamente
\warpHTMLonly{\makeatletter

% lwarp emula comandos de floatrow, mas não os comandos de float
% que existem em floatrow; aqui, incluímos a emulação de float
\usepackage{float}

% Este comando de floatrow não é emulado, mas nós utilizamos,
% então emulamos aqui
\providecommand{\DeclareFloatVCode}[2]{}

% Por padrão ele faz "blockquotation", que pelo jeito não existe no HTML.
% Funciona no browser porque ele cria uma regra CSS correspondente, mas
% o pandoc e o libreoffice, ao importar, ignoram a formatação. Aqui,
% repetimos o código de lwarp para "quote".
\renewenvironment*{quotation}
  {
    \LWR@forcenewpage
    \LWR@htmlblocktag{blockquote}
  }
  {\LWR@htmlblocktag{/blockquote}}

% Modifica verse em lwarp para permitir o asterisco em \\*
\renewenvironment{verse}
  {
    \newcommand{\@LWRNELnewline}{\@ifstar{\newline}{\newline}}
    \let\\\@LWRNELnewline
    \list{}{\itemsep      \z@
            \itemindent   -1.5em%
            \listparindent\itemindent
            \rightmargin  \leftmargin
            \advance\leftmargin 1.5em}%
    \item\relax}
  {\endlist}

% Hyperlinks de/para as notas de rodapé
% O identificador das notas de rodapé nos apêndices e nos anexos é
% o mesmo (...fn.A-1, onde A é o "número" do apêndice ou anexo).
% Seria bom acrescentar \@chapapp ou algo assim para contornar isso.
\ifcsdef{chapter}
  {
    \long\def\@makefntext#1{%
        \hypertarget{fn.\thechapter-\@thefnmark}{}%
	\hyperlink{fnbk.\thechapter-\@thefnmark}{\textsuperscript{\@thefnmark}}%
        ~#1%
    }

    \def\@makefnmark{%
        \hypertarget{fnbk.\thechapter-\@thefnmark}{}%
	\hyperlink{fn.\thechapter-\@thefnmark}{\textsuperscript{\@thefnmark}}%
    }
  }
  {
    \long\def\@makefntext#1{%
        \hypertarget{fn.\thesection-\@thefnmark}{}%
	\hyperlink{fnbk.\thesection-\@thefnmark}{\textsuperscript{\@thefnmark}}%
        ~#1%
    }

    \def\@makefnmark{%
        \hypertarget{fnbk.\thesection-\@thefnmark}{}%
	\hyperlink{fn.\thesection-\@thefnmark}{\textsuperscript{\@thefnmark}}%
    }
  }

% lwarp defines \quad as html code "&#x2003;". If "FormatWP" is true,
% \hspace is defined as a sequence of \quad's that approximate the
% dimension requested. If this dimension is large, however, the text
% generated (&#x2003;&#x2003;&#x2003;...) may be too large to fit
% in a line when the "fake" pdf is generated by lwarp (it becomes
% an overfull hbox). If this happens, the html output becomes incorrect.
% We would like LaTeX to be able to break the line, but for that it
% needs some space characters. We cannot simply add spaces or line
% breaks between successive \quad's because this would modify the size
% of the \hspace. What we do, then, is add comments: <!-- -->. But
% since we also do not want to generate an enormous amount of lines
% (because this might make the text longer than the pdf page), we add
% a comment like this only every 7 \quads. Note that we might add
% such a comment in places where there are no sequence of \quad's,
% but this is simple and causes no significant problems. We might
% change this slightly so that we only do this inside \hspace.

\newcounter{quads}
\appto{\quad}{%
  \ifnumgreater{\thequads}{7}{\setcounter{quads}{0}<!-- -->}{\stepcounter{quads}}%
}

\makeatother
}
% Modificações sobre lwarp depois que ela implantou
% suas próprias modificações a todas as demais packages,
% carregadas no final do preâmbulo
\AtBeginDocument{\warpHTMLonly{\makeatletter

% O comando precisa existir para podermos fazer \renewcommand; como
% \poemtitle é definido por titlesec e lwarp desabilita todos os
% comandos de titlesec, precisamos fazer o comando existir.
%\providecommand{\poemtitle}{\relax}
%\renewcommand{\poemtitle}{\@ifstar{\realpoemtitle}{\realpoemtitle}}

%\newcommand{\realpoemtitle}[1]{\hspace{1.5em}\textbf{#1}\par\vspace{\parskip}}
%\renewcommand{\poemlist}[1]{\relax}

% lwarp's microtype emulation does something weird with this command;
% we cannot load lwarp-letterspace because it clashes with microtype;
% so, we just correct this here.
\DeclareDocumentCommand{\textls}{o +m}{#2}

\newlength\MKwidth%
\newlength\MKheight%
\renewcommand{\makecell}[2][]{%
\settowidth{\MKwidth}{#2}%
\parbox[#1]{\MKwidth}{#2}%
}

% listings: lwarp faz parskip=0, mas isso faz a primeira linha
% (onde vai o <pre>) ser escrita em cima da primeira linha do programa
% no pdf; com isso, o texto é extraído erroneamente. Aqui, acrescentamos
% uma linha
\xpatchcmd{\LWR@atbeginverbatim}{\LWR@orignewline}{\LWR@orignewline\LWR@orignewline}{}{}

% lwarp insere quebras de linha ao longo do texto (html) gerado para
% garantir que não haja overflow na vertical. Normalmente isso não é
% um problema, mas com verbatim (e, portanto, com listings), pode
% acontecer isto:
%
% <pre>
%     linha 1
% ----> quebra de página
%     linha 2
% ----> quebra de página
%     linha 3
% </pre>
%
% O problema é que, na segunda página, não há nada impresso além da
% linha 2; por conta disso, pdftotext não "percebe" que essa linha
% está indentada, o que faz código-fonte com listings aparecer com
% erros de indentação. O certo seria sempre "gobble" os espaços,
% mas aparentemente a reimplementação de lwarp para listings
% ignora esse parâmetro. Então, não sei como resolver isso.

% Outro problema com listings é que a numeração das linhas de
% código processa não as linhas de entrada, mas as linhas de
% saída, ou seja, o código html. Como esse código inclui alguns
% comentários (<!-- ... -->), há linhas a mais, gerando números
% errados. Para piorar, alguns desses números ficam dentro do
% comentário html, então o resultado visual é que há "pulos"
% nos números de linha. Também não sei como resolver, mas nem
% comecei a olhar as possibilidades.

% Não precisamos de ruledcaption; aqui, fazemos o comando simplesmente
% chamar o float program.
\renewenvironment{ruledcaption}[2][]{
    \begin{program}
        \ifstrempty{#1}
            {\caption{#2}}%
            {\caption[#1]{#2}}%
}
{\end{program}}

\makeatother
}}

% Algumas packages "padrão" da AMS, que são praticamente obrigatórias.
% Algumas delas devem ser carregadas antes de unicode-math ou das
% definições das fontes do documento.
\usepackage{amssymb}
\usepackage{amsthm}
\usepackage{amsmath}

% "fontenc" é um parâmetro do NFSS (sistema de gestão de fontes do
% LaTeX; consulte "texdoc fntguide" e "texdoc fontenc"). O default
% é OT1, mas ele tem algumas limitações; a mais importante é que,
% com ele, palavras acentuadas não podem ser hifenizadas. Por
% conta disso, quase todos os documentos LaTeX utilizam o fontenc
% T1. A escolha do fontenc tem consequências para as fontes que
% podem ser usadas com NFSS; hoje em dia T1 tem mais opções de
% qualidade, então não se perde nada em usá-lo. A package fontspec
% (para gestão de fontes usando outro mecanismo, compatível apenas
% com lualatex e xelatex) carrega fontenc automaticamente, mas
% usando outra codificação ("TU" e não "T1"). Ainda assim, é útil
% carregar o fontenc T1 (antes de carregar fontspec!) para o caso
% de alguma fonte "antiga" ser utilizada no documento (embora isso
% não seja recomendado: lualatex e xelatex só são capazes de
% hifenizar palavras acentuadas com o fontenc TU).
\usepackage[T1]{fontenc}

\ifunicodeengine
  % Não é preciso carregar inputenc com LuaTeX e XeTeX, pois
  % com eles utf8 é obrigatório.
  \usepackage{fontspec}

  % Ao invés de usar o sistema tradicional de LaTeX para gerir
  % as fontes matemáticas, utiliza as extensões matemáticas do
  % formato otf definidas pela microsoft. Ao ativar esta package
  % o mecanismo tradicional não funciona mais! Há poucas fontes
  % com suporte a unicode-math.
  \usepackage{unicode-math}
\else
  % O texto está escrito em utf8.
  \usepackage[utf8]{inputenc}

  % Permitem utilizar small caps + itálico (e outras pequenas
  % melhorias). Em geral, desnecessário com fontspec, a menos
  % que alguma package utilize especificamente. Algumas raras
  % packages de fontes podem causar conflitos com fontaxes, em
  % geral por utilizarem a package "concorrente" nfssext-cfr.
  \usepackage{fontaxes}
  \usepackage{mweights}
\fi

% Acesso a símbolos adicionais, como \textrightarrow, \texteuro etc.,
% disponíveis na maioria das fontes através do fontenc TS1 ou mudando
% momentaneamente para computer modern/latin modern. Raramente útil
% com lualatex/xelatex, mas não causa problemas. Várias packages de
% fontes carregam textcomp, às vezes com opções específicas; assim,
% para evitar problemas, vamos carregá-la no final do preâmbulo para
% o caso de ela não ter sido carregada antes.
\AtBeginDocument{\usepackage{textcomp}}

% microajustes no tamanho das letras, espaçamento etc. para melhorar
% a qualidade visual do resultado. LaTeX tradicional não dá suporte a
% nenhum tipo de microajuste; pdfLaTeX dá suporte a todos. LuaLaTeX
% e XeLaTeX dão suporte a alguns:
%
% * expansion não funciona com XeLaTeX
% * tracking não funciona com XeLaTeX; é possível obter o mesmo resultado
%   com a opção "LetterSpace" do pacote fontspec, mas a configuração é
%   totalmente manual. Por padrão, aumenta o afastamento entre caracteres
%   nas fontes "small caps"; o resultado não se presta ao uso na
%   bibliografia ou citações, então melhor desabilitar.
% * kerning e spacing só funcionam com pdfLaTex; ambas são funções
%   consideradas experimentais e nem sempre produzem resultados vantajosos.

\newcommand\microtypeopts{
  protrusion=true,
  tracking=false,
  kerning=false,
  spacing=false
}

% TeXLive 2018 inclui a versão 2.7a da package microtype e a versão
% 1.07 de luatex. Essa combinação faz aparecer um bug:
% https://tex.stackexchange.com/questions/476740/microtype-error-with-lualatex-attempt-to-call-field-warning-a-nil-value
% Aqui, aplicamos a solução sugerida, que não tem "contra-indicações".
\ifluatex
  \usepackage{luatexbase}
\fi

\ifxetex
  \usepackage[expansion=false,\microtypeopts]{microtype}
\else
  \usepackage[expansion=true,\microtypeopts]{microtype}
\fi

% Alguns "truques" (sujos?) para minimizar over/underfull boxes.
%
% Para fazer um texto justificado, é preciso modificar o tamanho dos espaços
% em cada linha para mais ou para menos em relação ao seu tamanho ideal. Para
% escolher as quebras de linha, TeX vai percorrendo o texto procurando lugares
% possíveis para quebrar as linhas considerando essa flexibilidade mas dentro
% de um certo limite mínimo/máximo. Nesse processo, ele associa a cada possível
% linha o valor *badness*, que é o nível de distorção do tamanho dos espaços
% daquela linha em relação ao ideal, e ignora opções que tenham badness muito
% grande (esse limite é dado por \tolerance). Depois de encontradas todas
% as possíveis quebras de linha e a badness de cada uma, TeX calcula as
% *penalties* das quebras encontradas, que são uma medida de quebras "ruins".
% Por exemplo, na configuração padrão, quebrar uma linha hifenizando uma
% palavra gera uma penalty de 50; já uma quebra que faça a última linha
% do parágrafo ficar sozinha na página seguinte gera uma penalty de 150.
% Finalmente, TeX calcula a "feiúra" de cada possível linha (demerits)
% com base na badness e nas penalties e escolhe a solução que minimiza os
% demerits totais do parágrafo. Os comandos \linebreak e \pagebreak funcionam
% simplesmente acrescentando uma penalty negativa ao lugar desejado para a
% quebra.
%
% Para cada fonte, o espaço em TeX tem um tamanho ideal, um tamanho mínimo e um
% tamanho máximo. TeX nunca reduz um espaço para menos que o mínimo da fonte,
% mas pode aumentá-lo para mais que o máximo. Se os espaços de uma linha ficam
% com o tamanho ideal, a badness da linha é 0; se o tamanho é
% reduzido/aumentado 50% do mínimo/máximo, a badness da linha é 12; se o
% tamanho é reduzido/aumentado para o mínimo/máximo, a badness é 100. Se esse
% aumento for de 30% além do máximo, a badness da linha é 200; se for de 45%
% além do máximo, a badness é 300; se for de 60% além do máximo, a badness é
% 400; se for de 100% além do máximo, a badness é 800. O valor máximo possível
% para badness é 10.000, que significa "badness infinita".
%
% \tolerance indica a badness máxima que TeX aceita para uma linha; seu valor
% default é 200. Assim, aumentar para, digamos, 300 ou 400, permite que
% TeX escolha parágrafos com maior variação no espaçamento entre as linhas.
% No entanto, no cálculo de demerits, a badness e as penalties de cada linha
% são elevadas ao quadrado, então TeX geralmente prefere escolher outras
% opções no lugar de uma linha ruim. Por exemplo, órfãs/viúvas têm demerit
% de 22.500 e dois hífens seguidos têm demerit de 10.000; já uma linha com
% badness 400 tem demerit 160.000. Portanto, não é surpreendente que a maioria
% dos parágrafos tenha demerits abaixo de 40.000, quase todos abaixo de 100.000
% e praticamente nenhum acima de 1.000.000. Isso significa que, para a grande
% maioria dos parágrafos, aumentar \tolerance não faz diferença: uma linha com
% badness 400 nunca será efetivamente escolhida se houver qualquer outra opção
% com badness menor. Também fica claro que não há muita diferença real entre
% definir \tolerance como 800 ou 9.999.
%
% O problema muda de figura se TeX não consegue encontrar uma solução. Isso
% pode acontecer em dois casos: (1) o parágrafo tem ao menos uma linha que não
% pode ser quebrada com badness < 10.000 ou (2) o parágrafo tem ao menos uma
% linha que não pode ser quebrada com badness < tolerance (mas essa badness é
% menor que 10.000).
%
% No primeiro caso, se houver várias possibilidades de linhas que não podem ser
% quebradas, TeX não vai ser capaz de compará-las e escolher a melhor: todas
% têm a badness máxima (10.000) e, portanto, a que gerar menos deméritos no
% restante do parágrafo será a escolhida. Na realidade, no entanto, essas
% linhas *não* são igualmente ruins entre si, o que pode levar TeX a fazer uma
% má escolha. Para evitar isso, TeX tenta novamente aplicando
% \emergencystretch, que "faz de conta" que o tamanho máximo ideal dos espaços
% da linha é maior que o definido na fonte. Isso reduz a badness de todas as
% linhas, o que soa parecido com aumentar \tolerance. Há três diferenças, no
% entanto: (1) essa mudança só afeta os parágrafos que falharam; (2) soluções
% que originalmente teriam badness = 10.000 (e, portanto, seriam vistas como
% equivalentes) podem ser avaliadas e comparadas entre si; e (3) como a badness
% de todas as linhas diminui, a possibilidade de outras linhas que
% originalmente tinham badness alta serem escolhidas aumenta. Esse último ponto
% significa que \emergencystretch pode fazer TeX escolher linhas mais
% espaçadas, fazendo o espaçamento do parágrafo inteiro aumentar e, portanto,
% tornando o resultado mais homogêneo mesmo com uma linha particularmente ruim.
%
% É esse último ponto que justifica o uso de \emergencystretch no segundo caso
% também: apenas aumentar a tolerância, nesse caso, poderia levar TeX a
% diagramar uma linha ruim em meio a um parágrafo bom, enquanto
% \emergencystretch pode fazer TeX aumentar o espaçamento de maneira geral no
% parágrafo, minimizando o contraste da linha problemática com as demais.
% Colocando a questão de outra maneira, aumentar \tolerance para lidar com
% esses parágrafos problemáticos pode fazê-los ter uma linha especialmente
% ruim, enquanto \emergencystretch pode dividir o erro entre várias linhas.
% Assim, definir \tolerance em torno de 800 parece razoável: no caso geral,
% não há diferença e, se um desses casos difíceis não pode ser resolvido com
% uma linha de badness até 800, \emergencystretch deve ser capaz de gerar um
% resultado igual ou melhor.
%
% Penalties & demerits: https://tex.stackexchange.com/a/51264
% Definições (fussy, sloppy etc.): https://tex.stackexchange.com/a/241355
% Mais definições (hfuzz, hbadness etc.): https://tex.stackexchange.com/a/50850
% Donald Arseneau defendendo o uso de \sloppy: https://groups.google.com/d/msg/comp.text.tex/Dhf0xxuQ66E/QTZ7aLYrdQUJ
% Artigo detalhado sobre \emergencystretch: https://www.tug.org/TUGboat/tb38-1/tb118wermuth.pdf
% Esse artigo me leva a crer que algo em torno de 1.5em é suficiente

\tolerance=800
\hyphenpenalty=100 % Default 50; se o texto é em 2 colunas, 50 é melhor
\setlength{\emergencystretch}{1.5em}

% Não gera warnings para Overfull menor que 1pt
\hfuzz=1pt
\vfuzz\hfuzz

% Não gera warnings para Underfull com badness < 1000
\hbadness=1000
\vbadness=1000

% Por padrão, o algoritmo LaTeX para textos não-justificados é (muito) ruim;
% este pacote implementa um algoritmo bem melhor
\usepackage[newcommands]{ragged2e}

% ragged2e funciona porque permite que LaTeX hifenize palavras em textos
% não-justificados quando necessário. No caso de textos centralizados,
% no entanto, isso em geral não é desejável. Assim, newcommands não é
% interessante para \centering e \begin{center}. newcommands também
% causa problemas com legendas se o float correspondente usa \centering
% (o que é muito comum). Assim, vamos voltar \centering e \begin{center}
% à definição padrão.
\let\centering\LaTeXcentering
\let\center\LaTeXcenter

% Com ragged2e e a opção "newcommands", textos curtos não-justificados
% podem gerar warnings sobre "underfull \hbox". Não há razão para pensar
% muito nesses warnings, então melhor desabilitá-los.
% https://tex.stackexchange.com/questions/17659/ragged2e-newcommands-option-produces-underfull-hbox-warnings
\makeatletter
\g@addto@macro{\raggedright}{\hbadness=\@M}
\g@addto@macro{\RaggedRight}{\hbadness=\@M}
\g@addto@macro{\raggedleft}{\hbadness=\@M}
\g@addto@macro{\RaggedLeft}{\hbadness=\@M}
\g@addto@macro{\flushleft}{\hbadness=\@M}
\g@addto@macro{\FlushLeft}{\hbadness=\@M}
\g@addto@macro{\flushright}{\hbadness=\@M}
\g@addto@macro{\FlushRight}{\hbadness=\@M}
\makeatother

% Espaçamento entre linhas configurável (\singlespacing, \onehalfspacing etc.)
\usepackage{setspace}

% LaTeX às vezes coloca notas de rodapé logo após o final do texto da
% página ao invés de no final da página; este pacote evita isso e faz
% notas de rodapé funcionarem corretamente em títulos de seções.
% Esta package deve ser carregada depois de setspace.
\usepackage[stable,bottom]{footmisc}

% Se uma página está vazia, não imprime número de página ou cabeçalho
\usepackage{emptypage}

% hyperref deve preferencialmente ser carregada próximo ao final
% do preâmbulo mas, para o caso de alguma package forçar a sua
% carga antes de executarmos \usepackage explicitamente, vamos
% garantir que estas opções estejam ativas.
\PassOptionsToPackage{
  unicode=true,
  pdfencoding=unicode,
  plainpages=false,
  pdfpagelabels,
  bookmarksopen=true,
  breaklinks=true,
  %hyperfootnotes=false, % polui desnecessariamente com bordercolor
}{hyperref}

% Carrega nomes de cores disponíveis (podem ser usados com hyperref e listings)
\usepackage[hyperref,svgnames,x11names,table]{xcolor}

% LaTeX define os comandos "MakeUppercase" e "MakeLowercase", mas eles têm
% algumas limitações; esta package define os comandos MakeTextUppercase e
% MakeTextLowercase que resolvem isso.
\usepackage{textcase}

% Em documentos frente-e-verso, LaTeX faz o final da página terminar sempre
% no mesmo lugar (exceto no final dos capítulos). Esse comportamento pode ser
% ativado explicitamente com o comando "\flushbottom". Mas se, por alguma
% razão, o volume de texto na página é "pequeno", essa página vai ter espaços
% verticais artificialmente grandes. Uma solução para esse problema é utilizar
% "\raggedbottom" (padrão em documentos que não são frente-e-verso): com essa
% opção, as páginas podem terminar em alturas ligeiramente diferentes. Outra
% opção é corrigir manualmente cada página problemática, por exemplo com o
% comando "\enlargethispage".
%\raggedbottom
\flushbottom

% Por padrão, LaTeX coloca uma espaço aumentado após sinais de pontuação;
% Isso não é tão bom quanto alguns TeX-eiros defendem :) .
% Esta opção desabilita isso e, consequentemente, evita problemas com
% "id est" (i.e.) e "exempli gratia" (e.g.)
\frenchspacing

% Trechos de texto "puro" (tabs, quebras de linha etc. não são modificados)
\usepackage{verbatim}

% Durante o processamento, LaTeX procura por arquivos adicionais necessários
% (tanto componentes do próprio LaTeX, como packages e fontes, quanto partes
% do conteúdo em si, como imagens carregadas com \includegraphics ou arquivos
% solicitados com \input ou \include) no diretório de instalação e também
% no diretório atual (ou seja, o diretório do projeto). Assim, normalmente
% é preciso usar caminhos relativos para incluir arquivos de subdiretórios:
% "\input{diretorio/arquivo}". No entanto, há duas limitações:
%
% 1. É necessário dizer "\input{diretorio/arquivo}" mesmo quando o arquivo
%    que contém esse comando já está dentro do subdiretório.
%
% 2. Isso não deve ser usado para packages ("\usepackage{diretorio/package}"),
%    embora na prática funcione.
%
% Há três maneiras recomendadas de resolver esses problemas:
%
% 1. Acrescentando os diretórios desejados ao arquivo texmf.cnf
%
% 2. Acrescentando os diretórios desejados às variáveis de ambiente
%    TEXINPUTS e BSTINPUTS
%
% 3. Colocando os arquivos adicionais na árvore TEXMF (geralmente, no
%    diretório texmf dentro do diretório do usuário).
%
% Essas soluções, no entanto, não podem ser automatizadas por este modelo
% e são um tanto complicadas para usuários menos experientes. Veja mais a
% respeito na seção 5 de "texdoc kpathsea" e em
% https://www.overleaf.com/learn/latex/Articles/An_introduction_to_Kpathsea_and_how_TeX_engines_search_for_files .
%
% A package import pode solucionar o primeiro problema, mas exige o uso
% de outro comando no lugar de \input, então não a usamos aqui.
%\usepackage{import}
%
% Uma solução mais simples é acrescentar os diretórios desejados à macro
% \input@path, originalmente criada para resolver um problema relacionado
% à portabilidade. Seu uso não é normalmente recomendado por razões de
% desempenho, mas no nosso caso (em que adicionamos apenas um diretório
% com poucos arquivos e com máquinas modernas) isso não é um problema. Veja
% https://tex.stackexchange.com/questions/241828/define-path-for-packages-in-the-latex-file-analog-of-inputpath-or-graphicspa#comment705011_241832
\csappto{input@path}{{extras/}}
